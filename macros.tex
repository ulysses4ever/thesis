%              Convenience macros
% ****************************************************************************************************
\newcommand{\HIDEFORLONGVERSION}[1]{}
\newenvironment{itquote}{\begin{quote}\itshape}{\end{quote}\ignorespacesafterend}

%% Formatting
\newcommand{\EM}[1]{\ensuremath{#1}\xspace}
\newcommand{\xt}[1]{{\mathsf{#1}}}
\newcommand{\bt}[1]{\xt{\bf #1}}
\newcommand{\EMxt}[1]{\EM{\xt{#1}}}

% struct type
\newcommand{\Struct}[3]  {\EM{\bt{struct}\;#1\,\bt{is}\,#2\,\bt{end}}}
\newcommand{\any}    {\EMxt{any}}  %% any type
\newcommand{\obj}    {\EMxt{obj}}  %% object type
\renewcommand{\P}{\EMxt{P}}           %% compiled program
\renewcommand{\S}{\EMxt{S}}           %% source program
\newcommand{\M}{\EMxt{M}}           %% method table
\newcommand{\Red}{\EM{\rightarrow}} %% reduction step
\newcommand{\Comp}{\EM{\Rightarrow}} %% compilation step
\newcommand{\m}   {\EMxt m}          %% method name
\newcommand{\emp}{\EM{\epsilon}}  %% empty


\DeclareDocumentCommand\TR{om}{\EM{\IfNoValueTF{#1}{\PackageWarning{}{Undefined Type System}}{#1}\llbracket #2 \rrbracket}}

\DeclareDocumentCommand\TRG{omm}{\EM{\IfNoValueTF{#1}{\PackageWarning{}{Undefined Type System}}{#1}\llbracket #2 \rrbracket_{#3}}}
\DeclareDocumentCommand\TAG{ommm}{\EM{\IfNoValueTF{#1}{\PackageWarning{}{Undefined Type System}}{#1}\llparenthesis #2 \rrparenthesis_{#3}^{#4}}}

\newcommand{\OTS}{{\mathcal{O}}}
\newcommand{\CTS}{{\mathcal{C}}}
\newcommand{\BTS}{{\mathcal{B}}}
\newcommand{\TTS}{{\mathcal{T}}}
\newcommand{\SOMS}{{\mathcal{S}}}
\newcommand{\sspce}{;~}
\newcommand{\idbody}[1]{\SubCast{#1}\x\sspce \SubCast{#1}\x}

\newcommand{\bscast}[2]{\EM{\BehCast{#1}{{#2}}}}
\newcommand{\kty}[1]{\EM{\xt{kty}(#1)}}

\newcommand{\WHERE}{~\EM{\xt{\bf where}}~}
\newcommand{\OR}{\EM{~\xt{\bf or}}~}
\newcommand{\IF}{\EM{~\xt{\bf if}}~}

\newcommand{\HS}{\hspace{.2cm}}
\newcommand{\LS}{\hspace{1cm}}
\newcommand{\namet}[2]{\EM{#1\!\!\,:\,\!#2}}
\newcommand{\TypeCk}[3]{\EM{#1\vdash #2:#3}}

%% Variables
\newcommand{\x}   {\EMxt x}
\newcommand{\xp}   {\EMxt{x'}}
\newcommand{\n}   {\EMxt n}

\renewcommand{\mp}   {\EMxt{m'}}
\newcommand{\s}   {\EM{\sigma}}
\DeclareDocumentCommand\a{o}{\IfNoValueTF{#1}{\EMxt {a}}{\EM{\xt {a}_{#1}}}}
\DeclareDocumentCommand\ap{o}{\IfNoValueTF{#1}{\EMxt {a'}}{\EM{\xt {a'}_{#1}}}}
\DeclareDocumentCommand\app{o}{\IfNoValueTF{#1}{\EMxt {a''}}{\EM{\xt {a''}_{#1}}}}
\DeclareDocumentCommand\t{o}{\IfNoValueTF{#1}{\EMxt t}{\EM{\xt t_{#1}}}}
\DeclareDocumentCommand{\tp}{o}{\IfNoValueTF{#1}{\EM{ \xt t' }}{\EM{\xt t_{#1}'}}}
\DeclareDocumentCommand{\tpp}{o}{\IfNoValueTF{#1}{\EM{ \xt t'' }}{\EM{\xt t_{#1}''}}}
\DeclareDocumentCommand{\tppp}{o}{\IfNoValueTF{#1}{\EM{ \xt t''' }}{\EM{\xt t_{#1}'''}}}
\DeclareDocumentCommand{\e}{o}{\IfNoValueTF{#1}{\EM{ \xt e }}{\EM{\xt e_{#1}}}}
\DeclareDocumentCommand{\ep}{o}{\IfNoValueTF{#1}{\EM{ \xt e' }}{\EM{\xt e_{#1}'}}}
\DeclareDocumentCommand{\epp}{o}{\IfNoValueTF{#1}{\EM{ \xt e'' }}{\EM{\xt e''_{#1}}}}
\DeclareDocumentCommand{\eppp}{o}{\IfNoValueTF{#1}{\EM{ \xt e''' }}{\EM{\xt e'''_{#1}}}}
\DeclareDocumentCommand{\fd}{o}{\IfNoValueTF{#1}{\EM{ \xt{fd} }}{\EM{\xt{fd}_{#1}}}}
\DeclareDocumentCommand{\fdp}{o}{\IfNoValueTF{#1}{\EM{ \xt{fd}' }}{\EM{\xt{fd}_{#1}'}}}
\DeclareDocumentCommand{\fdpp}{o}{\IfNoValueTF{#1}{\EM{ \xt{fd}'' }}{\EM{\xt{fd}_{#1}''}}}
\DeclareDocumentCommand{\fdppp}{o}{\IfNoValueTF{#1}{\EM{ \xt{fd}''' }}{\EM{\xt{fd}_{#1}'''}}}
\DeclareDocumentCommand{\md}{o}{\IfNoValueTF{#1}{\EM{ \xt{md} }}{\EM{\xt{md}_{#1}}}}
\DeclareDocumentCommand{\f}{o}{\IfNoValueTF{#1}{\EM{ \xt f }}{\EM{\xt f_{#1}}}}
\DeclareDocumentCommand{\mdp}{o}{\IfNoValueTF{#1}{\EM{ \xt{md}' }}{\EM{\xt{md}_{#1}'}}}
\DeclareDocumentCommand{\mdpp}{o}{\IfNoValueTF{#1}{\EM{ \xt{md}'' }}{\EM{\xt{md}_{#1}''}}}
\DeclareDocumentCommand{\mdppp}{o}{\IfNoValueTF{#1}{\EM{ \xt{md}''' }}{\EM{\xt{md}_{#1}'''}}}
\DeclareDocumentCommand{\C}{o}{\IfNoValueTF{#1}{\EM{ \xt{C} }}{\EM{\xt{C}_{#1}}}}
\DeclareDocumentCommand{\mt}{o}{\IfNoValueTF{#1}{\EM{ \xt{mt} }}{\EM{\xt{mt}_{#1}}}}
\DeclareDocumentCommand{\mtp}{o}{\IfNoValueTF{#1}{\EM{ \xt{mt}' }}{\EM{\xt{mt}_{#1}'}}}
\DeclareDocumentCommand{\mtpp}{o}{\IfNoValueTF{#1}{\EM{ \xt{mt}'' }}{\EM{\xt{mt}_{#1}''}}}
\DeclareDocumentCommand{\mtppp}{o}{\IfNoValueTF{#1}{\EM{ \xt{mt}''' }}{\EM{\xt{mt}_{#1}'''}}}
\DeclareDocumentCommand{\D}{o}{\IfNoValueTF{#1}{\EM{ \xt{D} }}{\EM{\xt{D}_{#1}}}}
\DeclareDocumentCommand{\Dp}{o}{\IfNoValueTF{#1}{\EM{ \xt{D'} }}{\EM{\xt{D'}_{#1}}}}
\DeclareDocumentCommand{\Dpp}{o}{\IfNoValueTF{#1}{\EM{ \xt{D''} }}{\EM{\xt{D''}_{#1}}}}


\newcommand{\K}   {\EMxt K}
\renewcommand{\k} {\EMxt k}
\newcommand{\Kk}   {\K~\k}
\newcommand{\Kp}  {{\EMxt{K'}}}
\newcommand{\Kpp}  {{\EMxt{K''}}}
\newcommand{\Kppp}  {{\EMxt{K'''}}}
\renewcommand{\sp}{{{\EM{\s'}}}}
\newcommand{\spp}{{{\EM{\s''}}}}
\newcommand{\A}   {\EMxt {A}}
\newcommand{\I}   {\EMxt {I}}
\newcommand{\E}   {\EMxt {E}}
\newcommand{\Cp}  {\EMxt{C'}}
\newcommand{\Cpp}  {\EMxt{C''}}
\newcommand{\Cppp}  {\EMxt{C'''}}
\newcommand{\cmd}  {\EMxt{M}}
\newcommand{\cmdp}  {\EMxt{M'}}
\newcommand{\Env}   {\EM{\Gamma}}
\newcommand{\Envp}   {\EM{\Gamma'}}
\newcommand{\EE}   {\EM{\textsf{{E}}}}
\newcommand{\this}{\EMxt{this}}
\newcommand{\that}{\EMxt{that}}
\newcommand{\none}{\EM{\cdot}}
\newcommand{\CW}    {\EMxt{?C}}
\newcommand{\CWp}   {\EMxt{?C'}}
\newcommand{\CWpp}  {\EMxt{?C''}}
\newcommand{\DW}     {\EMxt{?D}}
\newcommand{\DWp}    {\EMxt{?D'}}
\newcommand{\DWpp}   {\EMxt{?D''}}


\newcommand{\FRead}[1]   {\EM{\this.#1}}
\newcommand{\FWrite}[2]  {\EM{\this.#1} = #2}
\newcommand{\FReadR}[2]   {\EM{#1.#2}}
\newcommand{\FWriteR}[3]  {\EM{#1.#2} = #3}
\newcommand{\Call}[3]  {\EM{#1.#2(#3)}}
\newcommand{\KCall}[5] {\EM{{#1}.{#2}_{{#4} \shortrightarrow {#5}}(#3)}}
\newcommand{\DynCall}[3]  {\EM{#1@#2_{\any\shortrightarrow\any}(#3)}}
\newcommand{\sDynCall}[3]  {\EM{#1@#2(#3)}}

\newcommand{\New}[2]   {\EM{\new\;#1(#2)}}
%\newcommand{\SubCast}[2]{\EM{<\hspace{-.6mm}{#1}\hspace{-.6mm}>\hspace{-1mm}\;{#2}}}
\newcommand{\SubCast}[2]{\EM{\langle{#1}\rangle\,{#2}}}
\newcommand{\BehStart}{\EM{\blacktriangleleft}}
\newcommand{\BehEnd}{\EM{\blacktriangleright}}
\newcommand{\BehCast}[2]{\EM{\BehStart\! #1\! \BehEnd #2}}

\newcommand{\new}      {\EM{\bt{new}}}
\newcommand{\HT}[2]    {\EM{{#1}\!:{#2}}}
\newcommand{\Mdef}[5]  {\EM{ \HT{ #1( \HT{#2}{#3})}{#4}\;\{{#5}\}}}
\newcommand{\ThorSub}[4]{\EM{#1~#2 \vdash #3 \Sub_t #4}}
\newcommand{\behcast}[7]{\EM{#5\,#6\,#7 = \xt{bcast}(#1, #2, #3, #4)}}
\newcommand{\behcastE}[7]{\EM{\xt{bcast}(#1, #2, #3, #4) = #5\,#6\,#7}}
\newcommand{\behcastS}[4]{\EM{\xt{bcast}(#1, #2, #3, #4)}}

\newcommand{\Alt}[1]   { &\B #1 \\}
\newcommand{\B}        {\EM{~|~}}

\newcommand{\Reduce}[6]   {\EM{{#1}~{#2}~{#3} \Red {#4}~{#5}~{#6}}}
\newcommand{\ReduceA}[6]  {\EM{#1~#2~#3 } & \EM{\Red #4~#5~#6}}
\newcommand{\class}       {\EM{\bf{class}}}
\newcommand{\Class}[3]    {\EM{\bt{class}\;#1\,\{\,#2~#3\,\}}}

\newcommand{\Fdef}[2]    {\EM{ \HT{#1}{#2} }}
\newcommand{\Mtype}[3]    {\EM{ \HT{#1(#2)}{#3}}}
\newcommand{\opdef}[2]    {\frmbox[1.1\width]{#1} ~ #2\\}
\newcommand{\Map}[2]     {\EM{ #1[#2] }}
\newcommand{\Bind}[2]     {\EM{#1 \mapsto #2}}

\newcommand{\Sub}{\EM{<:}}
\newcommand{\OK}{\EM{~\checkmark}}
\newcommand{\names}[1]{\EM{\xt{names}(#1)}}
\newcommand{\cload}[1]{\EM{\xt{nodups}(#1)}}

\newcommand{\ConsSub}{\EM{\lesssim}}

\newcommand{\EvalRulePassthrough}[1]{#1}
\newcommand{\EvalRuleLabel}[1]{\EvalRulePassthrough{\tiny\scshape\textcolor{gray}{#1}}}
\newcommand{\CondRule}[3]{ \EvalRuleLabel{#1} & #3 & #2 \\}
\newcommand{\SuchRule}[3]{ #3 &~{\emph{s.t.}} #2 \\}
\newcommand{\EnvType}[5]{ \EM{#1\,#2\,#3\vdash #4 : #5}}
\newcommand{\EnvTypex}[5]{ \EM{#1\,#2\,#3\vdash_{\!s} #4 : #5}}

\newcommand{\RuleRef}[1]{\hyperlink{infer:#1}{\TirNameStyle{#1}}}
\newcommand{\IRule}[4][]{\inferrule*[lab={\tiny \hypertarget{infer:#2}{#2}},#1]{#3}{#4}}
\newcommand{\Rule}[4][]{\inferrule*{#3}{#4}}
\newcommand{\HasType}[3]{ \EM{#1} (\EM{#2}) = \EM{#3}}
\newcommand{\wrap}[4]{\EM{\xt{W}(#1,#2,#3,#4)}}
\newcommand{\wrapE}[1]{\EM{\xt{W}(#1)}}
\newcommand{\wrapAny}[3]{\EM{\xt{W}\!\any(#1,#2,#3)}}

\newcommand{\classoff}[2]{\EM{\xt{mtypes}(#1,~#2)}}

\newcommand{\mtype}[3]{\EM{\xt{mtype}(#1,#2,#3)}}

\newcommand{\Convertible}[3]{\EM{#1 \vdash #2 \Mapsto #3}}
\newcommand{\ConvertE}[4]{\EM{#1 \vdash_{\!s} #3 \Mapsto #4}}
\newcommand{\In}{\EM{\in}}
\newcommand{\T}{\EM{\xt T}}
\newcommand{\AND}{\EM{\wedge}}
\newcommand{\App}[2]{\EM{#1(#2)}}

\newcommand{\SSub}[4]{\EM{#1~#2\vdash_{\!s} #3\Sub #4}}
\newcommand{\StrSub}[4]{\EM{#1~#2\vdash #3\Sub #4}}
\newcommand{\StrNotSub}[4]{\EM{#1~#2\vdash #3 \not\Sub #4}}
\newcommand{\ThrSub}[4]{\EM{#1~#2\vdash_{\!s} #3~\src\Sub~#4}}
\newcommand{\ConSub}[4]{\EM{#1~#2 \vdash #3 \lesssim #4}}
\newcommand{\OKW}{\EM{~\checkmark_{s}}}
\newcommand{\OKX}[1]{\EM{~\checkmark_{#1}}}
\newcommand{\EnvTypeW}[4]{ \EM{#1\,#2\vdash_{\!s} #3 : #4}}
\newcommand{\EnvTypeS}[4]{ \EM{#1\,#2\vdash_{\!s} #3 : #4}}
\newcommand{\EnvTypeT}[4]{ \EM{#1\,#2\vdash_{\!s} #3 : #4}}
\newcommand{\EnvTypeE}[5]{ \EM{#1\,#2\vdash_{\!s} #4 : #5}}
\newcommand{\trulename}[1]{#1}

\newcommand{\sign}[1]{\xt{signature}(#1)}

\newcommand{\WFtype}[2]{\EM{#1\vdash#2 \OK}}
\newcommand{\WF}[4]{\EM{#1\,#2\,#3\vdash#4 \OK}}
\newcommand{\WFp}[2]{#1~#2\OK}
\newcommand{\WFq}[1]{#1\OK}

\newcommand{\WFpx}[2]{#1~#2\OK_x}
\newcommand{\WFx}[4]{\EM{#1\,#2\,#3\vdash_{\!s}~#4 \OK}}
\newcommand{\WFX}[5]{\EM{#1\,#2\,#3\vdash_{\!#5}~#4 \OK}}
\newcommand{\WFtypex}[2]{\EM{#1\vdash_{\!s}~ #2 \OK}}
\newcommand{\WFtypeX}[3]{\EM{#1\vdash_{\!s}~ #3 \OK}}

\newcommand{\WFtypeW}[2]{\EM{#1\vdash_{\!s}~ #2 \OK}}
\newcommand{\WFW}[3]{\EM{#1\,#2\vdash_{\!s}~#3 \OK}}
\newcommand{\WFpW}[2]{#1~#2\OKW}
\newcommand{\WFpX}[3]{\EM{#1~#2\OKX{#3}}}

\newcommand{\V}{\EM{\checkmark}}

\newcommand{\dyn}[1]{\xt{dyn}(#1)}

\newcommand{\fresh}[1]{\EM{#1~\xt{fresh}}}
\newcommand{\Kt}[1]{\EM{\text{ktype}(#1)}}
\newcommand{\All}[1]{\EM{\forall ~\xt #1 ~.~}}
\newcommand{\SP}{\hspace{.5cm}}
\newcommand{\SPP}{\SP\SP}

\newcommand{\kafka}{{\mathsf{KafKa}}\xspace}
\newcommand{\src}[1]{\colorbox[gray]{0.89}{$#1$}}
\newcommand{\dt}[1]{\,\xt{?}#1}
\newcommand{\consistent}[3]{\EM{#1 \vdash #2 ~\sim~ #3}}
\newcommand{\rtranst}[6]{#1 \Rightarrow #2 ~ #3 / #4 \vdash #5 \looparrowright_{beh} #6}
\newcommand{\rtranstz}[4]{#1 \Rightarrow #2 \vdash #3 \looparrowright_{mon} #4}


\makeatletter
\newcommand{\efqn}{
\foreach\n in {1,...,\@listdepth}{foo}
}
\makeatother
\newlist{myEnumerate}{enumerate}{10}

\newenvironment{proofy}{
  \begin{myEnumerate}[resume=proofsteps,label={\textbf{\arabic*}. },ref=\arabic*]
}{
  \end{myEnumerate}
}
\DeclareDocumentCommand\stepp{o}{\IfNoValueTF{#1}{\item}{\item\hypertarget{proofstep:#1}{}\label{proofstep:#1}}}

\newcommand{\basis}{\hfill}
\newcommand{\caseof}[1]{\item\text{Case: } #1}
\newcommand{\rcaseof}[1]{\item\text{Case: } \RuleRef{#1}}

\newcommand{\refby}[1]{%
  \def\nextitem{\def\nextitem{,}}% Separator
  \renewcommand*{\do}[1]{\nextitem\hyperlink{proofstep:##1}{\ref{proofstep:##1}}}% How to process each item
  (\docsvlist{#1})% Process list
}
\newenvironment{casel}{
  \begin{myEnumerate}[label={\Alph*. }]
}{
  \end{myEnumerate}
}
\newenvironment{iknown}{
  \begin{enumerate*}[series=proofsteps,label={\textbf{\arabic*}. },ref=\arabic*]
}{
  \end{enumerate*}
}
%%%%%%%%%%%%%%%%%%%%%%%%%%%%%%%%%%%%%%%%%%%%%%%%%%%%%%%%%%%%%%%%%%%%%%%%%%%%%%%%%%%%%%%%%%%%%%%%%%%%%%%%%%%%%%%%%%%%%%%%%%%%%%%%%%%%%%%%%5

\newcommand{\greyBox}[1]{\colorbox[gray]{0.89}{$#1$}}

% Math definitions.
\newcommand{\ebox}[1]{\fbox{#1}\hfill\vspace{-1em}\centering}
\newcommand{\ex}[2]{\exists#1.~#2}
\newcommand{\fa}[2]{\forall#1.~#2}
\newcommand{\crd}[1]{\left|\ol{#1}\right|}

% Environment definitions.
\newcommand{\nothing}{\cdot}
\newcommand{\ctx}[2]{#1\vdash#2}
\newcommand{\ctxol}[2]{\ol{#1\vdash#2}}
\newcommand{\tjdg}[3]{\ctx{#1}{#2:#3}}
\newcommand{\gtjdg}[2]{\tjdg{\Gamma}{#1}{#2}}
\newcommand{\ctjdg}[3]{\tjdg{\Gamma,#1}{#2}{#3}}
\newcommand{\etjdg}[2]{\tjdg{\nothing}{#1}{#2}}
\newcommand{\tjdgol}[3]{\ctx{#1}{\ol{#2:#3}}}
\newcommand{\gtjdgol}[2]{\tjdgol{\Gamma}{#1}{#2}}
\newcommand{\ctjdgol}[3]{\tjdgol{\Gamma,#1}{#2}{#3}}
\newcommand{\etjdgol}[2]{\tjdgol{\nothing}{#1}{#2}}

\newcounter{rules}
\newenvironment{rules}[2]{\vspace{0.5em}\ebox{#2}
  \vspace{-1.5em}\setlength{\parskip}{2em}
  \renewenvironment{rule}[1]{%
    \protected@edef\@currentlabel{\textsc{#1-##1}}
    \RightLabel{(\textsc{#1-##1})}
  }{}
}{\vspace{1em}}

% Multiplicities with overline.
\newcommand{\ol}[1]{\EM{\overline{#1}}}

% Proof summaries.
\newcounter{theo}
\renewcommand{\thetheo}{\textbf{\arabic{theo}}}
\newcommand{\theo}[1]
  {\vspace{0.5em}\noindent\refstepcounter{theo}\textbf{Theorem}~\thetheo~(#1).}
\newcounter{lem}
\renewcommand{\thelem}{\textbf{\arabic{lem}}}
\newcommand{\lem}[1]
  {\vspace{0.5em}\noindent\refstepcounter{lem}\textbf{Lemma}~(#1).}


\newcommand{\listindent}{\addtolength\leftskip{2em}}

\newenvironment{analysis}[1]
  {\vspace{.5em} \textbf{Case analysis} on #1\par
    \renewcommand{\thestep}{\textbf{\roman{step}}}}{}

\newcommand{\stepwidth}{.55\textwidth-\leftskip}
\newcommand{\numwidth}{3.5em}

\newcommand{\numbox}[2]{\refstepcounter{#1}\makebox[\numwidth][l]{#2.}}
\newcommand{\stepnum}{\numbox{step}{\thestep}}

\newcommand{\statem}[1]{\stepspace\stepnum #1}
\newcommand{\sletsingle}[1]{\stepspace\stepnum\textbf{let}~#1}
\newcommand{\slet}[2]{\stepspace\stepnum\textbf{let}~#1~=~#2}
\newcommand{\slets}[3]{\stepspace\stepnum\makebox[\stepwidth][l]{\textbf{let}~#1~=~#2} by #3}
\newcommand{\sletwhere}[3]{\stepspace\stepnum\textbf{let}~#1~=~#2 \textbf{where} #3}
\newbool{stepspace}
\newcommand{\stepspace}
  {\ifbool{stepspace}{\vspace{0.4em}\boolfalse{stepspace}}{}\par}
\newcommand{\step}[2]
  {\stepspace\par\stepnum\makebox[\stepwidth][l]{#1} by #2}
\newcommand{\longstep}[2]
  {\stepspace\par\stepnum\makebox[\stepwidth][l]{#1} \par\makebox[9.4cm+\numwidth + \stepwidth] {by #2}}
\newcommand{\longstepA}[2]
  {\stepspace\par\stepnum\makebox[\stepwidth][l]{#1} \par\makebox[11.65cm+\numwidth + \stepwidth] {by #2}}
\newcommand{\longstepAA}[2]
  {\stepspace\par\stepnum\makebox[\stepwidth][l]{#1} \par\makebox[10.2cm+\numwidth + \stepwidth] {by #2}}
\newcommand{\longstepAAA}[2]
  {\stepspace\par\stepnum\makebox[\stepwidth][l]{#1} \par\makebox[14.80cm+\numwidth + \stepwidth] {by #2}}
\newcommand{\longstepB}[2]
  {\stepspace\par\stepnum\makebox[\stepwidth][l]{#1} \par\makebox[12.25cm+\numwidth + \stepwidth] {by #2}}
\newcommand{\longstepC}[2]
  {\stepspace\par\stepnum\makebox[\stepwidth][l]{#1} \par\makebox[11.75cm+\numwidth + \stepwidth] {by #2}}
\newcommand{\longstepD}[2]
  {\stepspace\par\stepnum\makebox[\stepwidth][l]{#1} \par\makebox[8.5cm+\numwidth + \stepwidth] {by #2}}
\newcommand{\longstepE}[2]
  {\stepspace\par\stepnum\makebox[\stepwidth][l]{#1} \par\makebox[8.75cm+\numwidth + \stepwidth] {by #2}}
\newcommand{\longstepF}[2]
  {\stepspace\par\stepnum\makebox[\stepwidth][l]{#1} \par\makebox[11.2cm+\numwidth + \stepwidth] {by #2}}
\newcommand{\longstepG}[2]
  {\stepspace\par\stepnum\makebox[\stepwidth][l]{#1} \par\makebox[11.2cm+\numwidth + \stepwidth] {by #2}}
\newcommand{\longstepH}[2]
  {\stepspace\par\stepnum\makebox[\stepwidth][l]{#1} \par\makebox[13.5cm+\numwidth + \stepwidth] {by #2}}
\newcommand{\longstepI}[2]
  {\stepspace\par\stepnum\makebox[\stepwidth][l]{#1} \par\makebox[11.3cm+\numwidth + \stepwidth] {by #2}}
\newcommand{\longstepII}[2]
  {\stepspace\par\stepnum\makebox[\stepwidth][l]{#1} \par\makebox[12.0cm+\numwidth + \stepwidth] {by #2}}
\newcommand{\longstepIII}[2]
  {\stepspace\par\stepnum\makebox[\stepwidth][l]{#1} \par\makebox[13.2cm+\numwidth + \stepwidth] {by #2}}
\newcommand{\longstepMJ}[2]
  {\stepspace\par\stepnum\makebox[\stepwidth][l]{#1} \par\makebox[11.0cm+\numwidth + \stepwidth] {by #2}}
\newcommand{\longstepMK}[2]
  {\stepspace\par\stepnum\makebox[\stepwidth][l]{#1} \par\makebox[11.7cm+\numwidth + \stepwidth] {by #2}}
\newcommand{\longstepML}[2]
  {\stepspace\par\stepnum\makebox[\stepwidth][l]{#1} \par\makebox[11.2cm+\numwidth + \stepwidth] {by #2}}
\newcommand{\longstepMM}[2]
  {\stepspace\par\stepnum\makebox[\stepwidth][l]{#1} \par\makebox[11.2cm+\numwidth + \stepwidth] {by #2}}
\newcommand{\longstepMN}[2]
  {\stepspace\par\stepnum\makebox[\stepwidth][l]{#1} \par\makebox[11.0cm+\numwidth + \stepwidth] {by #2}}
\newcommand{\longstepMO}[2]
  {\stepspace\par\stepnum\makebox[\stepwidth][l]{#1} \par\makebox[11.2cm+\numwidth + \stepwidth] {by #2}}

\newsavebox{\stepsby}
\newenvironment{steps}[1]
  {\savebox{\stepsby}{#1}
    \renewcommand{\step}[1] {\stepnum##1\par}\stepspace
    \begin{math}\left.\hspace{-.1em}\begin{minipage}{\numwidth+\stepwidth}}
  { \end{minipage}\right\}\end{math} by \usebox{\stepsby}\par}

\newenvironment{longsteps}[1]
  {\newcommand{\stepsbysavedvalue}{#1}
    \renewcommand{\step}[1] {\stepnum##1\par}\stepspace
    \begin{math}\left.\hspace{-.1em}\begin{minipage}{\numwidth+\stepwidth}}
  { \end{minipage}\right\}\end{math} \begin{minipage}{\textwidth-\stepwidth+1em}by \stepsbysavedvalue{}\end{minipage}\par}

% \newenvironment{for}[2]
%  {\vspace{.5em}For all~#1$~\in~$#2\par\listindent\stepcounter{steps}
%    \renewcommand{\thestep}{\textbf{\roman{step}}}\booltrue{stepspace}}
%  {\par\booltrue{stepspace}\stepcounter{steps}}

\newcommand{\qs}{\quad\quad}
\newcommand{\na}{case N/A}
\newcommand{\trivial}{\vspace{.4em}Trivial.}
\newcommand{\trivialind}
  {\vspace{.4em}Result follows directly from the induction hypothesis.}
\newcommand{\done}[1]{\step{done}{#1}}
\newcommand{\byeq}[1]{\ref{eq:#1}}
\newcommand{\byind}{ind hyp}
\newcommand{\bycontra}{contradiction}
\newcommand{\bysubst}{substitution}
\newcommand{\byrewrite}{Barendregt}
\newcommand{\byprem}[1]{prem~\ref{eq:#1}}
\newcommand{\byprems}[1]{prems~\ref{eq:#1}}
\newcommand{\bylem}[1]{lemma~\ref{lem:#1}}
\newcommand{\bylems}[2]{lemmas~\ref{lem:#1},~\ref{lem:#2}}
\newcommand{\byth}[1]{theorem~\ref{th:#1}}
\newcommand{\bythh}[2]{theorems~\ref{th:#1},~\ref{th:#2}}
\newcommand{\byfun}[1]{\fname{#1}}

\newcommand{\bypre}[2]{the premise of \texttt{\sc #1} on #2}
\newcommand{\bydef}[2]{the definition of \texttt{\sc #1} on #2}
\newcommand{\bydefE}[3]{the definition of \texttt{\sc #1} on #2 with #3}
\newcommand{\byjdg}[2]{\texttt{\sc #1} on #2}
\newcommand{\bysub}[1]{substitution convention on #1}
\newcommand{\bypres}[2]{the premises of \texttt{\sc #1} on #2}
\newcommand{\bylemm}[3]{\texttt{\sc lemma #1} on #3}
\newcommand{\bylemms}[3]{\texttt{\sc lemma #1:} #2 on #3}
\newcommand{\byweaklemm}[3]{\texttt{\sc Weakening lemma #1} on #3}
\newcommand{\byweaklemms}[3]{\texttt{\sc Weakening lemma #1:} #2 on #3}
\newcommand{\byindhyp}[1]{the inductive hyp on #1}
\newcommand{\bycons}[1]{the contradiction assumption on #1}
\newcommand{\bycontrad}[1]{contradicting statements #1}
\newcommand{\bysyn}[1]{syntax convention on #1}
\newcommand{\bysubs}[1]{substitution convention of #1}

\newcommand{\indmsg}[1]{induction on the derivation of #1}
\newcommand{\ind}[1]{by straightforward \indmsg{#1}.}
\newcommand{\indana}[1]{by \indmsg{#1}, with a case analysis on the last step: \\}
\newcommand{\indanaa}[1]{by \indmsg{#1}, with a case analysis on the last step. \\}
\newcommand{\strind}[1]{by structural induction on #1:}
\newcommand{\innat}{by natural deduction.\\}
\newcommand{\innatE}[1]{by natural deduction #1.\\}
\newcommand{\contrad}{by contradiction.\\}

\newcommand{\dyninsidesubmany}[3]{\ensuremath{\heap \vdash #1 (\overline{#2 \preceq #3})}}
\newcommand{\dyninsidesubmanyC}[3]{\ensuremath{#1 \vdash \overline{#2 \preceq #3}}}
\newcommand{\dyninsidesubmanyE}[5]{\ensuremath{#1 \vdash #2( #3 (\overline{#4 \preceq #5}))}}
\newcommand{\dyninsidesubmanyA}[4]{\ensuremath{#1 \vdash #2 (\overline{#3 \preceq #4})}}
\newcommand{\dyninsidesubmanyB}[3]{\ensuremath{#1 \vdash \overline{#2 \preceq #3}}}
\newcommand{\oks}{\mbox{ \sc ok}}

\ProvidesFile{omscmtt.fd}
\DeclareFontFamily{OMS}{cmtt}{\skewchar\font48 }
\DeclareFontShape{OMS}{cmtt}{m}{n}%
   {<->ssub*cmsy/m/n}{}
\DeclareFontShape{OMS}{cmtt}{m}{it}%
   {<->ssub*cmsy/m/n}{}
\DeclareFontShape{OMS}{cmtt}{m}{sl}%
   {<->ssub*cmsy/m/n}{}
\DeclareFontShape{OMS}{cmtt}{m}{sc}%
   {<->ssub*cmsy/m/n}{}
\DeclareFontShape{OMS}{cmtt}{bx}{n}%
   {<->ssub*cmsy/b/n}{}
\DeclareFontShape{OMS}{cmtt}{bx}{it}%
   {<->ssub*cmsy/b/n}{}
\DeclareFontShape{OMS}{cmtt}{bx}{sl}%
   {<->ssub*cmsy/b/n}{}
\DeclareFontShape{OMS}{cmtt}{bx}{sc}%
   {<->ssub*cmsy/b/n}{}

\newcommand{\IGNOREUNLESSNEEDED}[1]{}
\newcommand{\figref}[1]{Fig.~\ref{#1}\xspace}
\newcommand{\lemref}[1]{Lem.~\ref{#1}\xspace}
\newcommand{\thmref}[1]{Thm.~\ref{#1}\xspace}
\newcommand{\ruleref}[1]{Rule~{\small #1}\xspace}
\newcommand{\chapref}[1]{Chap.~\ref{#1}\xspace}
\newcommand{\secref}[1]{Sec.~\ref{#1}\xspace}
\newcommand{\ssecref}[1]{Subsec.~\ref{#1}\xspace}
\newcommand{\defref}[1]{Def.~\ref{#1}\xspace}
\newcommand{\appref}[1]{Appendix~\ref{#1}\xspace}

\newcommand{\wraps}[2]{#1\hspace{-0.2em}\Rightarrow\hspace{-0.2em}#2}
\renewcommand{\EM}[1]{\ensuremath{#1}\xspace}   %% make sure we are in math mode
\newcommand{\alt}{~\vert~}
\newcommand{\SF}[1]{\mathsf{#1}}                %% plain font for code elements in math
\newcommand{\SC}[1]{\textsc{#1}}
\newcommand{\tinyb}[1]{\scalebox{0.8}{{\normalsize #1}}}  %%% small
\renewcommand{\v}[1]{\EM{{\tinyb{\%}}\SF{#1}}}  %% variables start with a %
\newcommand{\tdef}[2]{\EM{#1\!::\!\SF{#2}}}     %% type definition have a ::
\newcommand{\val}{\EM{\SF{v}}}                  %% value
\newcommand{\tv}[2]{\EM{\tdef{\v{#1}}{#2}}}     %%  a typed variable
\newcommand{\ass}[2]{\EM{\v{#1} \leftarrow #2}} %%% assignment
\newcommand{\mdef}[3]{\EM{\tdef{\SF{#1}(#2)}{#3}}}  %%% method definition name(args) rettype
\renewcommand{\int}{\EM{\SF{Int}}}              %% integers
\renewcommand{\any}{\EM{\SF{Any}}}              %% any
\newcommand{\ty}{\EM{\SF{ty}}}                   %% type
\newcommand{\aty}{\EM{\SF{A}}}                  %% abstract type
\newcommand{\tdecl}{\EM{\SF{d}}}                %% type declaration
\renewcommand{\dyn}{\EM{*}}                     %% dyn
\newcommand{\call}[2]{\EM{\SF{#1}(#2)}}         %% function call
\newcommand{\cond}[4]{\EM{\ass{#1}{\v{#2} ~?~ #3 : #4}}}  %%% conditional assignment
\newcommand{\get}[2]{\EM{\v{#1}[\SF{#2}]}}      %% field read
\renewcommand{\i}{\EM{\SF{i}}}                  %% integer
\renewcommand{\j}{\EM{\SF{j}}}                  %% integer
\renewcommand{\m}{\EM{\SF{m}}}                  %% integer
\renewcommand{\k}{\EM{\SF{k}}}                  %% integer
\newcommand{\p}{\EM{\SF{p}}}                    %% primitive
\renewcommand{\l}{\EM{\SF{l}}}                  %% integer
\newcommand{\st}{\EM{\SF{st}}}                  %% statement
\newcommand{\env}{\EM{\SF{E}}}                  %% environment
\newcommand{\frm}{\EM{\SF{F}}}                  %% frame
\newcommand{\mtbl}{\EM{\SF{M}}}                 %% method table
\newcommand{\tytbl}{\EM{\SF{D}}}                %% type declarations table
\newcommand{\config}[2]{\EM{#1,\, #2}}          %% configuration
\newcommand{\configd}{\config{\frm}{\mtbl}}   %% default configuration F, M
\newcommand{\stk}[2]{\EM{#1 \cdot #2}}          %% stack
\renewcommand{\step}{\EM{~\rightarrow~}}        %% one step of reduction
\newcommand{\stepmul}{\EM{~\rightarrow^*~}}   %% multiple steps of reduction
%\newcommand{\stepd}{\EM{~\overset{{\mathcal{D}}}{\rightarrow}~}} %% one step of normal reduction
%\newcommand{\stepj}{\EM{~\overset{{\SF{jit}}}{\rightarrow}~}} %% one step of reduction with JIT
\newcommand{\stepd}{\EM{~{\rightarrow_{\mathcal{D}}}~}} %% one step of normal reduction
\newcommand{\stepj}{\EM{~{\rightarrow_{\SF{JIT}}}~}} %% one step of reduction with JIT
\newcommand{\stepdmul}{\EM{~{\rightarrow^*_{\mathcal{D}}}~}}
\newcommand{\stepjmul}{\EM{~{\rightarrow^*_{\SF{JIT}}}~}}
\newcommand{\idx}[2]{\EM{#1[{\SF{#2}}]}}        %% x[y]
%\renewcommand{\read}[3]{\EM{\mathit{fieldof}(#1,#2,\SF{#3})}} %%%
\newcommand{\construct}[2]{\EM{#1(#2)}}   %% allocating an object
\newcommand{\msig}[2]{\EM{#1!{#2}}}         %% method name with type annotation
\newcommand{\meth}[3]{\EM{#1!{#2}\,}}
\newcommand{\direct}[3]{\EM{\msig{#1}{#2}(#3)}}      %% direct call
\newcommand{\Ty}{\EM{\SF{T}}}                   %% type names
\newcommand{\last}[1]{\EM{\mathit{last}(#1)}}   %% last element of env
\renewcommand{\done}{\EM{\epsilon}}
\newcommand{\jules}{\EM{\SF{Jules}}}
\renewcommand{\c}[1]{\lstinline{#1}\xspace}
\newcommand{\VD}{\vdash}
%\newcommand{\VDnd}{\EM{\vdash^{\cancel{\mathcal{D}}}}}
\newcommand{\VDnd}{\EM{\vdash^{{\mathcal{D}}}}}
\newcommand{\VDm}{\vdash_\mtbl}
\newcommand{\VDo}{\EM{\vdash^{\mathrm{O}}}}
\renewcommand{\n}{\EM{\SF n}}
\newcommand{\nn}{\EM{\n+1}}
\newcommand{\err}{\EM{\SF{\mathbf{err}}}}       %% dispatch error
\newcommand{\main}{\EM{\SF{main}}}
\newcommand{\origmtbl}[1]{\EM{\lfloor #1 \rfloor}}
%% compilation
%\newcommand{\compst}[7]{\EM{#1\ \VD\ \config{#2}{\config{#3}{#4}}\ \leadsto\
%  \config{#5}{\config{#6}{#7}}}}
\newcommand{\compst}[7]{\EM{#1\ \VD\ \config{#2}{#3}\ \leadsto\
  \config{#5}{#6}}}
%% maximal devirtualization
\newcommand{\devirtst}[3]{\EM{#1\ \VDnd_{#2}\ #3}}
\newcommand{\devirtm}[1]{\EM{\VDnd\ #1}}
%% valid optimization
\newcommand{\optvalidst}[3]{\EM{#1\ \VDo_{#2}\ #3}}
\newcommand{\optvalidmtbl}[1]{\EM{\VDo #1}}
%% symbol |> for directed equivalence (instance of)
\newcommand{\eqdirop}{\triangleright}
%\newcommand{\eqdirop}{\approx}
% equivalence relation for instructions \ty.. |-_{M|>M'} st |> st'
\newcommand{\eqst}[5]{\EM{#1\ \vdash_{#2,#3}\ #4\ \eqdirop\ #5}}
\newcommand{\eqstd}[3]{\eqst{#1}{\mtbl}{\mtbl'}{#2}{#3}}
% equivalence relation for tables |- M |> M'
\newcommand{\eqmtbl}[2]{\EM{#1 \eqdirop #2}}
\newcommand{\eqmtbld}{\eqmtbl{\mtbl}{\mtbl'}}
% equivalence relation for configurations
\newcommand{\eqconfig}[2]{\EM{#1\ \eqdirop\ #2}}
%
% type inferred statement
\newcommand{\typeinfst}[3]{\EM{\VD^{\typeinfop}_{#1}\,#2\,<:\,#3}}
\newcommand{\typeinfstd}[2]{\typeinfst{\mtbl}{#1}{#2}}
%
% table, env, sts, tys
\newcommand{\frmtyped}[4]{\EM{\VD_{#1}\ #2\ #3 <: #4}}
% table, table', arg tys, sts, opt sts, sts tys
\newcommand{\optst}[5]{\EM{#3\,\VD_{#1,#2}\, #4\ \eqdirop\ #5}}
\newcommand{\optstd}[3]{\optst{\mtbl}{\mtbl'}{#1}{#2}{#3}}
\newcommand{\optframe}[6]{\optst{#1}{#2}{#3}{#4}{#5\,<:\,#6}}
\newcommand{\optframed}[4]{\optframe{\mtbl}{\mtbl'}{#1}{#2}{#3}{#4}}
\newcommand{\optconfig}[5]{\EM{\config{#1}{#2}\ \eqdirop\ \config{#3}{#4}\ <:\ #5}}
%
\DeclareMathOperator{\typeof}{\mathit{typeof}}
\DeclareMathOperator{\applicable}{\mathit{applcbl}}
\DeclareMathOperator{\dispatchop}{{\mathcal D}}
\DeclareMathOperator{\isconcrete}{\mathit{is-concrete}}
\DeclareMathOperator{\rettype}{\mathit{rettype}}
\DeclareMathOperator{\body}{\mathit{body}}
\DeclareMathOperator{\signature}{\mathit{signtr}}
\DeclareMathOperator{\origsignature}{\mathit{o-signtr}}
\DeclareMathOperator{\checkargs}{\mathit{checkargs}}
\DeclareMathOperator{\typeinfop}{\mathcal{I}}
\DeclareMathOperator{\jitop}{\mathit{jit}}
\newcommand{\dispatch}[3]{\EM{\dispatchop(#1,\SF{#2},{#3})}} %% dispatched call
\newcommand{\typeinf}[3]{\EM{\typeinfop(#1, #2, #3)}} %% type inference
\newcommand{\typeinfd}[1]{\typeinf{\mtbl}{\Gamma}{#1}} %% type inference
\newcommand{\jit}[0]{\EM{\jitop}} %% jitting
\newcommand{\TODO}[1]{\textcolor{red}{\textbf{TODO:} #1}}
%\newcommand{\TODO}[1]{}

\newtheorem{property}{Property}[section]
\newtheorem{requirement}{Requirement}[section]
\newtheorem{definition}{Definition}[chapter]

%  \theoremstyle{acmplain}
  \newtheorem{theorem}{Theorem}[chapter]
  \newtheorem{conjecture}{Conjecture}[chapter]
  \newtheorem{proposition}[theorem]{Proposition}
  \newtheorem{lemma}[theorem]{Lemma}
  \newtheorem{corollary}[theorem]{Corollary}
%  \theoremstyle{acmdefinition}
  \newtheorem{example}[theorem]{Example}
%  \newtheorem{definition}[theorem]{Definition}

\newcommand{\goodpkgsnum}{760\xspace}
\newcommand{\juliaversion}{Julia~1.5.4\xspace}

%% PAPER VERSION AND HIGHLIGHTING
%\usepackage{showframe}  %% for checking margins
\newcommand{\ExtendedVersion}{1}    %% 1 if yes, 0 if not
\newcommand{\HightlightChanges}{0}  %% 1 if yes, 0 if not

% For submission, use #1; for extended version, use #2
\newcommand{\PAPERVERSION}[2]{%1
\ifnum1=\ExtendedVersion\relax
#2%
\else
#1\xspace
\fi}

\newcommand{\PAPERVERSIONINLINE}[2]{%1
\ifnum1=\ExtendedVersion\relax
#2%
\else
#1%
\fi}

\newcommand{\ADD}[1]{%
\ifnum1=\HightlightChanges\relax
\cbcolor{green}
\begin{changebar}
#1%
\end{changebar}
\cbcolor{gray}
\else
#1%
\fi}

\newcommand{\MODIFY}[1]{%
\ifnum1=\HightlightChanges\relax
\begin{changebar}
#1
\end{changebar}
\else
#1
\fi}

% Compat with acmart
\newcommand{\Description}[1]{}

% Steps of algorithm
\newcommand{\textastep}[1]{%
\spacedlowsmallcaps{\textbf{#1}}%
}
\newcommand{\astep}[1]{%
\textastep{Step #1}\xspace%
}

% ATTENTION, passengers:
%     this is the VERY END of the file as pdflatex will see it!
\endinput
