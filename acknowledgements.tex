\cleardoublepage%
\phantomsection%
\chapter*{Acknowledgements}
\addcontentsline{toc}{chapter}{Acknowledgements}

This work would not be possible without… many people, actually! It is astonishing how many people were instrumental in getting where I am now. I cannot name everyone. But I will try my best.

My advisor Jan Vitek introduced me to the Julia programming language and, later on, type stability. For someone working only with statically typed languages since high school, this was a blast, but a manageable one (Jan has interests in languages way crazier than Julia). Jan also took me as a student with no clear idea about what to work on and with a notably distant interests from his own, which, in my view, is a great advance. He broadened my understanding of what programming languages research can be — probably the greatest scientific influence I ever had.

I came to Northeastern after spending 13 years in I.I. Vorovich Institute of Mathematics, Mechanics and Computer Science (MMCS), Southern Federal University (Rostov-on-Don, Russia) — first, as a student, and then, as a teaching faculty. Several people there left a lasting impact on my academic interests, but many more helped to create a sense of belonging and a safe space where I could pursue my teaching and research aspirations. I cannot express how much I enjoyed each day during those 13 years.

My BSc/MSc advisor, late Vladimir Deundyak, spent many hours with me talking about error-correcting codes (topic of my MSc thesis), mathematics (his true passion despite going undercover as a security researcher), academia, and politics. If I had to choose one quote from him that taught me the most, that would be: ``A mathematician has very few joys in life, so we should run them to the ground’’. And by ``mathematician’’ he really meant ``scientist’’.

Stanislav Mikhalkovich has been working on what I consider the largest (at least, in terms of the user base) compiler project in Russia — PascalABC.NET. It is a compiler and an IDE for teaching programming in a .NET-enabled and radically modernized dialect of Pascal — a language historically dominating the Russian educational landscape. Besides providing deep expertise in compilers, Stanislav shared and encouraged curiosity about programming language theory, which led several of us to organize a reading group on John C. Mitchell’s Foundations for Programming Languages. Stanislav also accepted me as a teacher in his ``Sunschool’’, a place where middle-school kids can learn about computers, digital creativity, and programming. A completely different perspective on computer science (and a critical source of income) for me over the years.

While at MMCS, I supervised a number of BSc and one MSc student. I still talk to some of them occasionally — they are all very special people to me not only because they learned something from me (I hope!), but also because I learned something from them in a unique way. Gosha Lukyanov, Volodya Pankov, Masha Vturina, Olya Filippskaya, Anya Bolotina, and others helped me feel better about what I am doing in academia. Gosha raised to earn a PhD in computer science from Newcastle U with Andrey Mokhov — his achievement I keep bragging about like I have anything to do with it! I am glad that Anya is on a similar track.

After I left MMCS in 2017, a ton of bright students and postdocs working with Jan helped merealize the wide field of programming languages. For creating a sense of a second professional family, I want to thank Filip Křikava, Peta Maj, Konrad Siek, Ryan Culpepper, Jan Jecmen. Jan joined me in building the prototype of the type stability inference tool and brought several deep insights (backed by pull requests) in how to improve it. The Czech lab and the year I spent in Prague are always in my heart.

PRL students during my tenure at NEU were all amazing, bright, and each taught me something. Starting from the old timers: Max, William, Ben G, Justin. And continuing with ``my cohort’’: Ellen, Alexi, Aviral, Hyeyoung, Celeste, Aaron, Ben C, Ming-Ho.

Ming-Ho Yee helped to improve my writing and presentations tremendously. He also took the lead in organizing social events in the lab when we needed them most — when the pandemic hit. Ming-Ho taught me that sharing your joys as well as grievances with colleagues may be appropriate and helpful to both parties in the conversation.

Mitch Wand, Emeritus Professor at the lab, taught an amazing course on technical writing that was taken over by the PRL crowd and turned into pure joy for the whole semester. If I learned one thing from Mitch, that would be the advice about putting an example in the introduction section of the paper to motivate the problem: when the paper comes to an end, the example should be solved one way or another. I keep seeing Mitch’s name on numerous old papers referenced byongoing research and cannot imagine what a great contribution he has made to our field.

My first lead co-author on a ``big’’ paper (OOPSLA ‘18) Francesco Zappa Nardelli showed me an example of endless perseverance while digging deep into the dark corners of Julia and its subtyping relation. Later on, he agreed to provide his time and highest expertise by serving on my thesis committee.

I owe my thesis committee a big thank you: Frank, Arjun, Francesco, and Jan were helpful and generous in sharing their insights about my work.

Besides my parents and close relatives, the two most influential people in my life are my wife Julia Belyakova and my best friend Vitaly Bragilevsky. Perhaps surprisingly, both of them not only shared a great amount of love, support, and care, but also influenced my research and work in academia.

I learned about lambda calculus in my second year of college, but it would remain a funny notation from the dusty folio of Barendregt with no real value if not for Vitaly, who helped me discover Haskell and programming languages theory. That was many years after he had helped me get geometry in 9th grade in my high school (``Liceum’’), and some time before he invited me to teach at MMCS. We share so many memories that my only worry is that, one day, I start forgetting some things. ``Memory Almost Full’’. But there is still space to add more.

I came to take for granted that Julia always sees the better in me. But she also actively helps me grow professionally and personally. I doubt I will ever be able to pay back for everything she gave me.

My mom and dad did a lot for me, but I want to specifically mention that I got the best education I can imagine given the circumstances because of them. Again, the debt is simply unimaginable.

I think my great grandma Nina who turned 98 three weeks before my defense beats everyone in the world in how much effort she has put into me. My single wish is to meet her in person again.

I am writing this while watching my soon-to-be 1-year-old daughter Sophia trying very hard to start walking. These past weeks she taught me about determination more than I ever knew. She is already a great teacher who I will keep learning from.
