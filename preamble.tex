% *********************************************************************************************
%
%              Core package
%
% *********************************************************************************************
\usepackage[
  drafting=false,           % print version information on the bottom of the pages
  linedheaders=true,        % chapter headers will have line above and beneath
  floatperchapter=true,     % numbering per chapter for all floats (i.e., Figure 1.1)
  %
  % TOC
  tocaligned=false,         % the left column of the toc will be aligned (no indentation)
  dottedtoc=true,           % page numbers in ToC flushed right
  %
  % Fonts
  eulerchapternumbers=true, % use AMS Euler for chapter font (otherwise Palatino)
  eulermath=true,           % use awesome Euler fonts for math (only with pdfLaTeX)
  beramono=true,            % toggle a nice monospaced font (w/ bold)
  palatino=true,            % deactivate standard font for loading another one,
                            %    see the last section at the end of this file for suggestions
  style=arsclassica       % classicthesis, arsclassica
  ]{classicthesis}

% arsclassica can't stomach long chapter names, so:
% https://tex.stackexchange.com/a/228223/7460
\renewcommand\formatchapter[1]{%
    \setbox0=\hbox{\chapterNumber\thechapter\hspace{10pt}\vline\ }%
    \begin{minipage}[t]{\dimexpr\linewidth-\wd0\relax}%
    \raggedright\spacedallcaps{#1}%
    \end{minipage}%
}

% \DeclareRobustCommand{\spacedallcaps}[1]{\textls[160]{\MakeTextUppercase{#1}}}
% \DeclareRobustCommand{\spacedlowsmallcaps}[1]{\textls[80]{\scshape\MakeTextLowercase{#1}}}

% *********************************************************************************************
%
%              Personal data
%
% *********************************************************************************************
\newcommand{\myTitle}{Type Stability in Julia}
\newcommand{\mySubtitle}{A Simple and Efficient Optimization Technique}
\newcommand{\myKws}{JIT, Julia, multiple dispatch, compiler optimizations, devirtualization}
\newcommand{\myDegree}{Ph.D.\xspace}
\newcommand{\myName}{Artem Pelenitsyn\xspace}
\newcommand{\myProf}{Jan Vitek\xspace}
\newcommand{\myDepartment}{Khoury College of Computer Sciences\xspace}
\newcommand{\myUni}{Northeastern University\xspace}
\newcommand{\myLocation}{Boston, USA\xspace}
\newcommand{\myTime}{2022\xspace}


% ********************************************************************
%
%        Various Convenience Packages
%
% ********************************************************************
\usepackage[utf8]{inputenc}
\usepackage[T1]{fontenc}
\usepackage{tcolorbox}
\usepackage{amsmath,amsthm}

\usepackage{xspace,url,hyperref,doi,wrapfig,stmaryrd,graphicx,xparse,etoolbox}
\usepackage{caption,subcaption}
\usepackage{floatrow} % floats (figure/table) are centered by default
\usepackage[inline]{enumitem}
\usepackage{cancel}
\usepackage[xcolor,rightbars]{changebar}
%\usepackage{showframe}  %% for checking margins
\definecolor{Gray}{gray}{0.9}
\definecolor{vlightgray}{gray}{0.93}


% ********************************************************************
%
%        Bibliography
%
% ********************************************************************
\usepackage[
  %backend=biber,bibencoding=utf8, %instead of bibtex
  backend=bibtex8,bibencoding=ascii,%
  language=auto,%
  style=numeric-comp,%
  %style=authoryear-comp, % Author 1999, 2010
  %bibstyle=authoryear,dashed=false, % dashed: substitute rep. author with ---
  sorting=nyt, % name, year, title
  maxbibnames=10, % default: 3, et al.
  %backref=true,%
  natbib=true % natbib compatibility mode (\citep and \citet still work)
]{biblatex}
\addbibresource{bib/jv.bib}
\addbibresource{bib/all.bib}
\addbibresource{bib/lj.bib}


% ********************************************************************
%
% Fine-tune hyperreferences (hyperref should be called last)
%
% ********************************************************************
\hypersetup{%
  %draft, % hyperref's draft mode, for printing see below
  colorlinks=true, linktocpage=true, pdfstartpage=3, pdfstartview=FitV,%
  % uncomment the following line if you want to have black links (e.g., for printing)
  %colorlinks=false, linktocpage=false, pdfstartpage=3, pdfstartview=FitV, pdfborder={0 0 0},%
  breaklinks=true, pageanchor=true,%
  pdfpagemode=UseNone, %
  % pdfpagemode=UseOutlines,%
  plainpages=false, bookmarksnumbered, bookmarksopen=true, bookmarksopenlevel=1,%
  hypertexnames=true, pdfhighlight=/O,%nesting=true,%frenchlinks,%
  urlcolor=CTurl, linkcolor=CTlink, citecolor=CTcitation, %pagecolor=RoyalBlue,%
  %urlcolor=Black, linkcolor=Black, citecolor=Black, %pagecolor=Black,%
  pdftitle={\myTitle},%
  pdfauthor={\myName},%
  pdfsubject={\myTitle},%
  pdfkeywords={\myKws},%
  pdfcreator={pdfLaTeX},%
  pdfproducer={LaTeX with hyperref and classicthesis}%
}

% *********************************************************************************************
%
%               Setup code listings
%
% *********************************************************************************************

\usepackage{style/julia}
\usepackage{mathpartir,listings}
\lstdefinelanguage{Jules}{
  keywords={struct,is,end},
  keywordstyle=\color{darkgray}\bfseries,
  ndkeywords={struct,is,end},
  ndkeywordstyle=\color{darkgray}\bfseries,
  identifierstyle=\color{black},
  sensitive=false,  comment=[l]{//},  morecomment=[s]{/*}{*/},
  commentstyle=\color{gray}\ttfamily,  stringstyle=\color{gray}\ttfamily,
  morestring=[b]',  morestring=[b]",
  aboveskip=\medskipamount, %0em,
  belowskip=\medskipamount, %0em
  escapeinside={(*@}{@*)}
}
\lstset{
  language=Jules,  extendedchars=true,  basicstyle=\small\ttfamily,
  showstringspaces=false,   showspaces=false,  numberstyle=\small,
  numbersep=9pt,  tabsize=2, breaklines=true,  showtabs=false, captionpos=b
}

\newcommand{\code}[1]{{\ttfamily #1}\xspace}
\renewcommand{\c}[1]{\lstinline[language=Julia]!#1!\xspace}

% *********************************************************************************************
%
%               Page Headings
%
% *********************************************************************************************

% \manualmark
% \markboth{\spacedlowsmallcaps{\contentsname}}{\spacedlowsmallcaps{\contentsname}}
\automark[section]{chapter}
% \renewcommand{\chaptermark}[1]{\markboth{\spacedlowsmallcaps{#1}}{\spacedlowsmallcaps{#1}}}
% \renewcommand{\sectionmark}[1]{\markright{\textsc{\thesection}\enspace\spacedlowsmallcaps{#1}}}
