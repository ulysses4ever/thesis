\chapter{Conclusions}%
\label{chap:conc}

In this dissertation I describe a program property called type stability and the
ways it is employed in the Julia programming language. To that end, I make
several contributions, which, hopefully enable better understanding of the
language itself and the general problem of utilization of run-time type
information in dynamic, just-in-time compiled languages.

First, in \chapref{chap:empirical} I show that the type stability property is
widely exercised in open source Julia packages. This finding may come as a
surprise given that there is no automated tools exist to check the property.
Perhaps, much of the Julia code is type stable because it is the most natural
way to express algorithms in the language. I show that certain ubiqituos code
patterns, e.g. type-constant functions or generic transformations, naturally
lead to type stable code. On the other hand, I point out certain code
patterns, especially those coming from traditional object-oriented languages,
that produce type unstable code. These are relatively well-known in the Julia
community and warned about in the language manual, which helps maintaining high
percentage of type stable code overall. Most encouraging is that Julia packages
aimed at performance-critical application have explicit notes about trying to
abide the property of type stability.

Second, my formal model of the Julia JIT in \chapref{chap:jules} helps to
pinpoint the relationship between type stability and runtime optimizations. The
Jules virtual machine recognizes code that I call type grounded and that can
only rely on type-stable APIs, and turns it in the most optimized version that I
call \emph{full devirtualization}. In practice, not every algorithm can be
easily coded in a type grounded fashion, so the property may be too demanding on
the first glance. My idea behind type groundedness is that it gives an ideal
that facilitates the argument for type stability. Also, as
\chapref{chap:empirical} shows, there is, actually, about half type grounded
code in an average Julia package.

Lastly, the formal model in \chapref{chap:jules} studies type stability on the
level of an intermediate representation inspired by Julia's own, but it is
unlikely to be an optimal model for a casual Julia programmer. That is why in
\chapref{chap:approx} I build an approach to understanding type stability in
terms of the source language and without running the program. The approach
reuses existing Julia tools like the type inferencer and implementation of
the subtyping relation. This idea of reusing key Julia components ensures that
we always agree with what the optimizer does at run time. My program analysis
can be run by a package author at the development time or as a part of their
continuous integration setup, and does not cost the end-user any performance.

\section{Future Work}

\paragraph{Evaluation on a large-scale application} The
evaluation of the type stability approximation approach has been done on a small
set of popular Julia packages. It is hard to predict how a package will be used,
and, by extension, what type instabilities will matter in practice. Therefore, a
better aim for the evaluation may be a Julia application that has a more clear
cost promise to the user. Currently, I am working with an industrial company
that builds such an application, which consists of about 200K lines of Julia
code. The application has a clear set of benchmark and a constant monitoring of
performance regressions. This may be a fruitful setting for tailoring the
approximation tool and ideas behind it (for example, the way types database
is assembled).

\paragraph{A tool for fixing type instabilities} The approach and tool described in
\chapref{chap:approx} can serve as a bug catching tool if ran on every
commit to a performance-oriented project to signal about any regression in type
stability of the code. Often times, a bug catching tool can be turned into a bug
fixing one. Indeed, there are several recipes in the Julia manual that help to
fix type instabilities but none of them were implemented as a tool, to the best
of my knowledge. A tool to fix type instabilities may be a good extension for
the current tool simply signaling about those instabilities.

\paragraph{Garbage code collection} While not directly connected to type
stability, another problem following from Julia's aggressive approach to type
specialization (besides sudden performance regressions due to type
instabilities) is code bloating. The formal model of the JIT in
\chapref{chap:jules} shows how the method table can grow indefinitely during
program execution. Although not an issue in the formal setting,  it can very
well lead to suboptimal performance even in type-stable code. A reasonable
extension of the current work relating performance and type stability would be a
study of performance implications of the type specialization strategy.
